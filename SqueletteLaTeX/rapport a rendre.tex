\documentclass{report}

\usepackage{a4wide}
\usepackage[utf8]{inputenc}
\usepackage[T1]{fontenc}
\usepackage[french]{babel}
\usepackage[babel=true]{csquotes} % guillemets français
\usepackage{graphicx}
\graphicspath{{Images/}}
\usepackage{color}
\usepackage{hyperref}
\hypersetup{colorlinks,linkcolor=,urlcolor=blue}

\usepackage{amsmath}
\usepackage{amssymb}


\title{Rapport programmation concurrente}
\author{\'SALVAN Fabien, L3 informatique}
\date{\today}

\begin{document}

\maketitle % pour écrire le titre


%% Le résumé:
\begin{abstract}
    Ce rapport comprend l'explication detaillée de chaque exercice du TP de programmation concurrente a rendre. Pour chaque exercice , les fonctions principales sont expliquées ainsi que l'architecture du programme.
\end{abstract}

\section{Introduction}
Dans le cadre du TP de programmation concurrente , on a eu a faire un exercice impliquant des balles , avec les consignes suivantes 
\begin{itemize}
\item Les balles ne peuvent sortir du cadre de la fenêtre et ricochent contre les bords
\item Un bouton permet de démarrer/arrêter le mouvement de toutes les balles. Lorsque les balles
sont en mouvement, ce bouton a pour étiquette “stop” et lorsque les balles sont à l’arrêt il
a pour étiquette “start”

\item Un bouton “+” permet d’ajouter une nouvelle balle de couleur aléatoire. Lorsqu’un seuil
fixé au préalable est atteint, plus aucune balle ne peut être ajoutée.

\item Un bouton “-” permet de supprimer une balle.

\item — Lorsqu’une collision se produit, les balles impliquées sont supprimées

\item Un score est affiché en permanence et est incrémenté à chaque collision.

\item Une horloge affiche le temps écoulé pendant que les balles sont en mouvement ; cette horloge
doit stopper son décompte lorque les balles sont arrêtées et reprendre son décompte lorsque
les balles sont à nouveau en mouvement.
  \end{itemize}


Les bibliotheques utilisées sont swing pour java et tkinter pour python
% \ref{...} permet de faire référence à un élément défini
% ailleurs dans le document (voir \label{...} plus haut).

\section{balles en java}
Pour faire cet exercice , il a fallu diviser le code en plusieurs classes , chacune ayant son role , et certaines etant executées en parralele afin de maximiser les performances.

\subsection{Fenetre}
 Fenetre : cette classe est derivée de la classe jframe de java , sert a ouvrir une fenetre pour faire le programme en affichage graphique

\subsection{Affiche}
 Affiche : Ce thread s'occupe juste de rafraichir en boucle la classe Fenetre , afin de pouvoir afficher les changement de positions des 
balles ainsi que l'UI . En changeant le temps de sleep de ce thread , on peut modifier la vitesse de la simulation.
\begin{verbatim}
			while(true){
				fenetre.repaint();
				sleep(1);
		}
\end{verbatim}

\subsection{Ball supervisor}

Ball supervisor : ce thread gere la physique des balles , c'est a dire il s'occupe en meme temps de leur deplacement , ainsi que des collisions
ce thread , comme tout les autres 
ce thread s'occupe de modifier la position des balles a chaque fois , en fonction
de leurs position et de leurs vitesse . Ilgere aussi les collisions.

\begin{verbatim}
			while(true){
				if(fenetre.pause == false){
				for(int i=0;i<fenetre.dessins.size();i++){
					fenetre.dessins.get(i).new_pos();
			}
				for(int i=0;i<fenetre.dessins.size();i++){
					for(int j=0;j<fenetre.dessins.size();j++){
						if(i !=j){
							if(fenetre.dessins.get(i).collision(fenetre.dessins.get(j))){
								fenetre.dessins.remove(j);
								fenetre.dessins.remove(i);
								fenetre.score_value +=1 ;
							}
						}
				}			}
				sleep(10);
		}
				else{
					sleep(10);
				}
\end{verbatim}

\subsection{Ball }

Ball : cette classe est utilisée pour instancier les balles . elle contient toutes les fonctions necessaire au fonctionnement des balles . Elle contient entre autres les fonctions utilisées par ball supervisor pour gerer la collision et la physique . 
\begin{itemize}
\item distance : calcule la distance de la balle avec une autre
\begin{verbatim}
	public double distance (Ball other){
		int dx = other.x - x;
		int dy = other.y - y;
		return Math.sqrt((dx*dx) + (dy*dy));
	}
\end{verbatim}
\item colision : determine si une autre balle est en colision ou pas
\begin{verbatim}
	public boolean collision(Ball other){
		if (((int)distance(other) <= (size+ other.size))){
			return true;
		}
		else{
			return false;
		}
\end{verbatim}
\item new pos : determine la prochaine position de la balle
\begin{verbatim}
	public void new_pos(){
		wall_colision();
		x = x+vx;
		y = y+vy;
	}
\end{verbatim}
\item ajout: ajoute une balle
\end{itemize}
\subsection{Timer}

Timer : thread utilisé pour faire fonctionner le chronometre de la fenetre. Il est lancé dans un thread a part afin que les autres thread n'interferent pas avec lui , de maniere a ne pas perdre en precision

\begin{verbatim}
			while(true){
				if(fenetre.pause == false){
					fenetre.time +=1;
					sleep(1000);
				}
				else{
					sleep(1000);
				}
\end{verbatim}



\subsection{Main}

 Main : classe pour executer le programme . le main est juste chargé de demarrer tout les threads neccessaires au bon fonctionnement du programme , et de créer une boucle infinie pour garder le programme ouvert

\begin{verbatim}
	public static void main(String[] args) {
	
		Fenetre f = new Fenetre("fenetre");
		Ball_supervisor s = new Ball_supervisor(f);
		Affiche a = new Affiche(f);
		Timer t = new Timer(f);
		a.start();s.start();
		t.start();
	}
\end{verbatim}



\begin{center}
  \includegraphics[scale=0.5]{Exo4.PNG}
\end{center}
et voici le resultat 

\subsection{points delicats/interessants }
\begin{itemize}
\item points delicats : il fallait reaprendre a programmer une fenetre en java , ce qui a necessité une certain temps . 
De plus , il a été assez delicat de coder la collision entre les balles  
\item points interessants :ca a été interessant d'arriver a programmer quelque chose en fenetre et de plus complexe que les exercices precedents.
Il a aussi été interessant de pouvoir créer un systeme de collision , ce qui permet de comprendre comment marchent les colisions en general dans 
des logiciels plus complexe

\end{itemize}




\section{balles en python}
\subsection{main }
La classe main permet d'afficher la fenetre , ainsi que de gerer l'UI. Elle contient les fonctions permettant de gerer l'UI de la fenetre

\begin{itemize}
\item pause : gere la pause
\item retrait : enleve une balle
\item close : ferme la fenetre

\item ajout: ajoute une balle
\end{itemize}

\subsection{ball }
Comme en java , la classe ball contient le code necessaire a instancier des balles . Les collisions entre les balles et les murs sont aussi
gerée dans cette classe  , grace a la fonction : 
\begin{verbatim}
	def collision(self,p):
		x=p.x-self.x
		y=p.y-self.y
		dist=x*x+y*y
		if (sqrt(dist))<=(taille):
			fenetre.nb-=2
			fenetre.nb_ball["text"]=("nombre de balle: {}".format(fenetre.nb))
			fenetre.score+=2
			fenetre.n_score["text"]=("score: {}".format(fenetre.score))
			fenetre.canvas.delete(self.name)
			ball.liste.remove(self)
			fenetre.canvas.delete(p.name)
			ball.liste.remove(p)
\end{verbatim}

\subsection{calcul }
Le thread calcul permet de gerer la physique , c'est a dire de deplacer les balles en fonction de leurs position et de leur vitesse



Et voici le resultat final : 

\begin{center}
  \includegraphics[scale=0.5]{ballepython.PNG}
\end{center}
\subsection{points delicats/interessants }
points delicats : il fallait reaprendre a programmer en python , etant donné que on en a fait que en 1ere année et avec des choses basiques , en plus des difficultées liées a la programmation en fenetre et en thread

\section{Conclusion}
En conclusion , ce cours a permis de bien comprendre les difficultées et les interet de la programmation par thread , ainsi que de pouvoir les mettre en application dans plusieurs languages de programmation . 

\section{Bibliographie}

%%% La bibliographie:
\bibliographystyle{plain}
\begin{itemize}
\item documentation java = {\url{http://www.oracle.com/technetwork/java/javase/documentation/index.html}}
\item documentation python = {\url {https://docs.python.org/3/}}

\end{itemize}



\end{document}
